\chapter{Introduction }
\label{cha:Introduction}

\section{FFG Project Description}
Industrial energy systems for manufacturing are mainly designed for single supply technologies, not designed for the fluctuation of energy demand and energy supply and thus can only react to a volatile demand and supply (thermal and electric) to a limited extent. From this, the need for the best possible support in optimizing the operation of the industrial energy system (demand and supply), the interaction of different renewable (volatile) and conventional energy sources and the design for industrial energy systems can be derived.
The demand for products from the printed circuit board industry is continuously on the rise. Besides the increase of production capacity, companies like AT\&S in Austria have to face the challenge of frequent change and adaptation to end-user requirements, causing significant changes in the energy demand and supply and by this, energy capacity limits on-site.This will be further increased by digitalisation and in terms of site security the need to increase productivity. The flexibility of the system makes it almost impossible for industry to plan and assess necessary adaptations and investment in the process and supply system and these challenges will increase significantly in the upcoming years.
The overall objective of DigitalEnergyTwin is to support the industry with the development of a methodology and software-tool to optimize the operation and design of industrial energy systems. The core of the project is the development of a  holistic optimization approach, based on (near-) real production data, historical and predictions of the existing system, both the process demand and supply level. By this, industry will be supported for the first time with reliable solutions in terms of fluctuating, volatile and renewable energy supply well designed for efficient process technologies. The methodology of the digital twin will be developed and validated for single use cases and more importantly implemented in the manufacturing industry (PCB industry). For selected processes (energy relevant) and renewable as well as conventional supply technologies also the product quality will be addressed within this approach.
Simplification and the development of technical standard solutions will lead to cost-effective exploitation in other industrial
Thus, the DigitalEnergyTwin builds on other areas of digitisation that are currently being developed. The use of the digital twin methodology will also make it possible to use the augmented and virtual reality (AR/VR) approach, which enables efficient production and system monitoring as well training and will support the EnergyManager 4.0 in the future. By this, a maximum impact and multiplication in other industrial companies and sectors will be achieved and the industry benefits from a reduction of costs and risk of investment decision, which will lead to a significant increase of the implementation of renewable energy technologies as well as technologies for higher energy efficiency in industrial production.
At the end of the project, DigitalEnergyTwin will provide the following main products and thus ensure that the gained knowledge and results will be used beyond the project lifetime: \\
• Holistic Optimization Algorithm and Software for industrial energy systems\\
• DigitalEnergyTwin software - Application of digital twin methodology to industrial energy systems \\
• Holistic and simplified energy models (process, energy supply conventional and renewable) \\
• Validation and standardized concept for data security, data management and handling between software and hardwarecomponents \\
• Standardized and simplified model and workflow for the multiplication of DigitalEnergyTwin \\
• Service provision and EnergyManager4.0 in combination with augmented and virtual reality (AR/VR) for human-to-machineinteraction in the context of industry 4.0.

\cite{FFGDET}

\section{•}

%\glsresetall %% all glossary entries should be used in long form (again)
%% vim:foldmethod=expr
%% vim:fde=getline(v\:lnum)=~'^%%%%\ .\\+'?'>1'\:'='
%%% Local Variables:
%%% mode: latex
%%% mode: auto-fill
%%% mode: flyspell
%%% eval: (ispell-change-dictionary "en_US")
%%% TeX-master: "main"
%%% End:
